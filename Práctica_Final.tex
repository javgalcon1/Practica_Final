% Options for packages loaded elsewhere
\PassOptionsToPackage{unicode}{hyperref}
\PassOptionsToPackage{hyphens}{url}
%
\documentclass[
]{article}
\usepackage{amsmath,amssymb}
\usepackage{lmodern}
\usepackage{iftex}
\ifPDFTeX
  \usepackage[T1]{fontenc}
  \usepackage[utf8]{inputenc}
  \usepackage{textcomp} % provide euro and other symbols
\else % if luatex or xetex
  \usepackage{unicode-math}
  \defaultfontfeatures{Scale=MatchLowercase}
  \defaultfontfeatures[\rmfamily]{Ligatures=TeX,Scale=1}
\fi
% Use upquote if available, for straight quotes in verbatim environments
\IfFileExists{upquote.sty}{\usepackage{upquote}}{}
\IfFileExists{microtype.sty}{% use microtype if available
  \usepackage[]{microtype}
  \UseMicrotypeSet[protrusion]{basicmath} % disable protrusion for tt fonts
}{}
\makeatletter
\@ifundefined{KOMAClassName}{% if non-KOMA class
  \IfFileExists{parskip.sty}{%
    \usepackage{parskip}
  }{% else
    \setlength{\parindent}{0pt}
    \setlength{\parskip}{6pt plus 2pt minus 1pt}}
}{% if KOMA class
  \KOMAoptions{parskip=half}}
\makeatother
\usepackage{xcolor}
\usepackage[margin=1in]{geometry}
\usepackage{color}
\usepackage{fancyvrb}
\newcommand{\VerbBar}{|}
\newcommand{\VERB}{\Verb[commandchars=\\\{\}]}
\DefineVerbatimEnvironment{Highlighting}{Verbatim}{commandchars=\\\{\}}
% Add ',fontsize=\small' for more characters per line
\usepackage{framed}
\definecolor{shadecolor}{RGB}{248,248,248}
\newenvironment{Shaded}{\begin{snugshade}}{\end{snugshade}}
\newcommand{\AlertTok}[1]{\textcolor[rgb]{0.94,0.16,0.16}{#1}}
\newcommand{\AnnotationTok}[1]{\textcolor[rgb]{0.56,0.35,0.01}{\textbf{\textit{#1}}}}
\newcommand{\AttributeTok}[1]{\textcolor[rgb]{0.77,0.63,0.00}{#1}}
\newcommand{\BaseNTok}[1]{\textcolor[rgb]{0.00,0.00,0.81}{#1}}
\newcommand{\BuiltInTok}[1]{#1}
\newcommand{\CharTok}[1]{\textcolor[rgb]{0.31,0.60,0.02}{#1}}
\newcommand{\CommentTok}[1]{\textcolor[rgb]{0.56,0.35,0.01}{\textit{#1}}}
\newcommand{\CommentVarTok}[1]{\textcolor[rgb]{0.56,0.35,0.01}{\textbf{\textit{#1}}}}
\newcommand{\ConstantTok}[1]{\textcolor[rgb]{0.00,0.00,0.00}{#1}}
\newcommand{\ControlFlowTok}[1]{\textcolor[rgb]{0.13,0.29,0.53}{\textbf{#1}}}
\newcommand{\DataTypeTok}[1]{\textcolor[rgb]{0.13,0.29,0.53}{#1}}
\newcommand{\DecValTok}[1]{\textcolor[rgb]{0.00,0.00,0.81}{#1}}
\newcommand{\DocumentationTok}[1]{\textcolor[rgb]{0.56,0.35,0.01}{\textbf{\textit{#1}}}}
\newcommand{\ErrorTok}[1]{\textcolor[rgb]{0.64,0.00,0.00}{\textbf{#1}}}
\newcommand{\ExtensionTok}[1]{#1}
\newcommand{\FloatTok}[1]{\textcolor[rgb]{0.00,0.00,0.81}{#1}}
\newcommand{\FunctionTok}[1]{\textcolor[rgb]{0.00,0.00,0.00}{#1}}
\newcommand{\ImportTok}[1]{#1}
\newcommand{\InformationTok}[1]{\textcolor[rgb]{0.56,0.35,0.01}{\textbf{\textit{#1}}}}
\newcommand{\KeywordTok}[1]{\textcolor[rgb]{0.13,0.29,0.53}{\textbf{#1}}}
\newcommand{\NormalTok}[1]{#1}
\newcommand{\OperatorTok}[1]{\textcolor[rgb]{0.81,0.36,0.00}{\textbf{#1}}}
\newcommand{\OtherTok}[1]{\textcolor[rgb]{0.56,0.35,0.01}{#1}}
\newcommand{\PreprocessorTok}[1]{\textcolor[rgb]{0.56,0.35,0.01}{\textit{#1}}}
\newcommand{\RegionMarkerTok}[1]{#1}
\newcommand{\SpecialCharTok}[1]{\textcolor[rgb]{0.00,0.00,0.00}{#1}}
\newcommand{\SpecialStringTok}[1]{\textcolor[rgb]{0.31,0.60,0.02}{#1}}
\newcommand{\StringTok}[1]{\textcolor[rgb]{0.31,0.60,0.02}{#1}}
\newcommand{\VariableTok}[1]{\textcolor[rgb]{0.00,0.00,0.00}{#1}}
\newcommand{\VerbatimStringTok}[1]{\textcolor[rgb]{0.31,0.60,0.02}{#1}}
\newcommand{\WarningTok}[1]{\textcolor[rgb]{0.56,0.35,0.01}{\textbf{\textit{#1}}}}
\usepackage{graphicx}
\makeatletter
\def\maxwidth{\ifdim\Gin@nat@width>\linewidth\linewidth\else\Gin@nat@width\fi}
\def\maxheight{\ifdim\Gin@nat@height>\textheight\textheight\else\Gin@nat@height\fi}
\makeatother
% Scale images if necessary, so that they will not overflow the page
% margins by default, and it is still possible to overwrite the defaults
% using explicit options in \includegraphics[width, height, ...]{}
\setkeys{Gin}{width=\maxwidth,height=\maxheight,keepaspectratio}
% Set default figure placement to htbp
\makeatletter
\def\fps@figure{htbp}
\makeatother
\setlength{\emergencystretch}{3em} % prevent overfull lines
\providecommand{\tightlist}{%
  \setlength{\itemsep}{0pt}\setlength{\parskip}{0pt}}
\setcounter{secnumdepth}{-\maxdimen} % remove section numbering
\ifLuaTeX
  \usepackage{selnolig}  % disable illegal ligatures
\fi
\IfFileExists{bookmark.sty}{\usepackage{bookmark}}{\usepackage{hyperref}}
\IfFileExists{xurl.sty}{\usepackage{xurl}}{} % add URL line breaks if available
\urlstyle{same} % disable monospaced font for URLs
\hypersetup{
  hidelinks,
  pdfcreator={LaTeX via pandoc}}

\author{}
\date{\vspace{-2.5em}}

\begin{document}

title: `Patrones de distribución de las Estelas decoradas de la
Península Ibérica' author: ``Javier Gallardo Contreras'' date:
`2023-05-31' output: pdf\_document: default ---

\hypertarget{introducciuxf3n}{%
\subsubsection{1. INTRODUCCIÓN:}\label{introducciuxf3n}}

La presencia de las conocidas como estelas decoradas, en todas sus
variantes, es un elemento distintivo, e incluso identitario, de la
compleja composición del paisaje ibérico, por su iconicidad y por el
gran simbolismo que desprenden. Mientras que las intervenciones a
pequeña escala en el medio paisajístico definieron a las sociedades
preneolíticas y/o no agrarias, el Neolítico (y posteriormente la Edad de
los Metales: Calcolítico, Bronce y Hierro) se caracterizó por un cambio
en la manera en que el ser humano concebía el paisaje, su organización y
su transformación (Carrero-Pazos, 2018). Esto se produjo, entre otras
cosas, a través de la construcción de una serie de monumentos que,
perpetuándose en el tiempo, vertebraron los territorios de las
sociedades vivas. Estas singulares manifestaciones artísticas comprenden
una horquilla cronológica que ha podido situarse desde principios del
Bronce (ca. 2200-850/825 AC) hasta la Edad del Hierro (ca. 850/825-700
AC) (Díaz-Guardaminos, 2010). Sin embargo, debemos aclarar que el
fenómeno de las estelas no es algo nuevo ni exclusivo del suroeste
peninsular. Con función y simbología general, estos monumentos llevan
existiendo en buena parte de la fachada atlántica europea, como mínimo
desde el Neolítico, si bien es cierto que la iconografía y detalles de
las estelas peninsulares las hacen únicas de esta región en las
cronologías previamente mencionadas. Sin ser nuestra pretensión
extendernos en exceso en este tema, sería ilustrativo realizar al menos
un acercamiento básico a los dos principales tipos en los que se
clasifican las estelas de la Prehistoria peninsular:

\begin{itemize}
\tightlist
\item
  Las más antiguas son las que se conocen como de formato básico,
  estelas cuya superficie es horadada hacia dentro, representando la
  propia piedra a una figura antropomorfa como si ella misma portase las
  armas en las zonas donde las sujetaría una persona, pero evitando
  crear una verdadera forma antropoide.
\item
  A partir de la llegada de los fenicios al sur de la Península Ibérica,
  comienzan a aparecer las estelas de formato antropomorfo, es decir, la
  piedra pasa a ser el soporte sobre el que se va a representar el
  sujeto, ya no recoge la ``esencia'' del personaje, sino que se graban
  figuras humanas. Desde el momento que aparecen este tipo de estelas,
  se enriquece la decoración con todo tipo de objetos que aparecerán
  representados sobre su superficie: serán frecuente las fíbulas,
  escudos, espadas, hachas, etc, elementos que funcionan como un buen
  marcador cronológico (Diaz-Guardamino, 2011). Podemos, a su vez.
  distinguir dos variantes de las estelas con antropomorfos:

  \begin{itemize}
  \tightlist
  \item
    Las estelas de guerrero, que representan a un individuo guerrero y
    su panoplia (espada, arco y flecha).
  \item
    Las estelas diademadas, que tienen como característica principal un
    arco a modo de diadema que rodea la cabeza del sujeto representado
    (el significado de este símbolo es aún objeto de debate, pues tiene
    varias interpretaciones: tocado, aureola, símbolo de
    feminidad\ldots).
  \item
    Las estelas grupales, en las que aparecen más de un individuo,
    recreando una escena en concreto y con varios elementos
    representados.
  \end{itemize}
\end{itemize}

Estos singulares monumentos han venido siendo objeto de estudio desde el
pasado siglo, pero muy especialmente en las últimas décadas (70-80, en
adelante). Sin embargo, y a pesar de ello, las interpretaciones que de
estas representaciones simbólico-artísticas se ha dado están aún en
desarrollo. Sin negar una posible significación funeraria-conmemorativa
(como remarcaron autores como Almagro-Bach en los años 60, haciendo
alusión a estas como ``estelas-guijarro'' o ``ídolos-estela''), la
creencia de que estos monumentos prehistóricos funcionasen como tumbas
fue una primera interpretación que actualmente, aunque continúa vigente,
apenas sí tiene cabida en el paradigma ``estelar'' de la cuestión. Por
su parte, la asociación de estelas con caminos, límites de fronteras
(geográficas o ecológicas) e hitos visibles, goza de mayor
plausibilidad, si bien su significado continúa siendo objeto de una
amplia controversia en el panorama epistemológico debido a la posible
multifuncionalidad que podrían encerrar. El estudio de las relaciones de
las estelas con el espacio que las rodea es crucial para comprender la
interrelación que se establece entre ellas y el medio ambiente en que se
localizan, es decir, analizar por igual la geografía y la cronología de
estos elementos. La aparición de las estelas en lugares significativos
de las redes terrestres que comunican las diferentes zonas del Suroeste
entre sí y con el exterior puede haber sido su función primordial'', y
que las representaciones en ellas grabadas ``contengan una información
codificada, accesible a quien se mueva por las vías donde fueron
situadas'' (Galán y Martín, 1993: 17). En este sentido, la existencia de
las estelas del suroeste como hitos significativos en los caminos pudo
deberse a lo imprescindible que resultan los marcadores del paisaje al
ahora de viajar, ya que ``sin ellos, la construcción de estas
representaciones mentales y la navegación terrestre serían imposibles''
(Murrieta et al, 2011). Así, y adentrándonos en nuestra materia de
estudio, es probable que los marcadores en el paisaje durante la
Prehistoria (es decir, los primeros marcadores utilizados) constituyesen
elementos naturales reconocibles y conspicuos. Siguiendo esta línea, los
monumentos megalíticos habrían de ser uno de estos marcadores
geográficos, que funcionarían además como manifestaciones intencionadas
de marcado carácter simbólico.

\hypertarget{objetivos}{%
\subsubsection{2. OBJETIVOS:}\label{objetivos}}

Una visión de conjunto de la distribución de las estelas del suroeste
(que en total suman 125 muestras, sirviéndonos del mapa de Celestino y
Paniego) permite comprobar que la mayor parte de ellas se encuentran
situadas en puntos estratégicos y singulares geográficamente hablando:
puertos de montaña, vados, zonas de contacto entre diferentes suelos,
resaltes orográficos, etc. La pregunta que debemos formularnos es la
siguiente: ¿responde la ubicación de estos elementos al más puro azar, o
de lo contrario sigue unos patrones concretos? El objetivo principal de
este trabajo es el de confirmar que efectivamente, la localización de
estos singulares monumentos obedecen a una pauta geográfica, que en
nuestro caso, y atendiendo a cartografías de autores como Sebastián
Celestino Pérez y Pablo Paniego (consultar Fig.1 en Celestino y Paniego,
2021: 73), estaría directamente relacionado con el relieve peninsular.
Es decir, la orografía sería el principal factor determinante que
explicaría la posición de las estelas decoradas, en detrimento de otras
teorías que interpretan las estelas como marcadores de tumbas, símbolos
de heroicidad o episodios bélicos.

\hypertarget{metodologuxeda}{%
\subsubsection{3. METODOLOGÍA:}\label{metodologuxeda}}

Expuesto nuestro razonamiento, debemos explicar cómo refutamos la teoría
de las estelas como marcadores geográficos y de territorialidad. Para
ello, hemos decidido hacer uso de la regresión lineal, un método
estadístico utilizado para modelar y analizar la relación entre una
variable dependiente (Y) y una o más variables independientes (X). El
objetivo de la regresión lineal es encontrar la mejor línea recta que se
ajuste a los datos para predecir valores futuros o comprender la
relación entre las variables. La regresión lineal asume que la relación
entre las variables puede describirse mediante una función lineal. La
ecuación de la regresión lineal tiene la forma:
\[Y = β₀ + β₁X₁ + β₂X₂ + ... + βₚXₚ + ε\] Donde:

\begin{itemize}
\tightlist
\item
  Y es la variable dependiente que deseamos predecir o explicar.
\item
  X₁, X₂, \ldots, Xₚ son las variables independientes que se utilizan
  para predecir o explicar Y.
\item
  β₀, β₁, β₂, \ldots, βₚ son los coeficientes que representan la
  contribución de cada variable independiente.
\item
  ε es el término de error, que representa la variabilidad no explicada
  por el modelo.
\end{itemize}

El objetivo de la regresión lineal es estimar los coeficientes β₀, β₁,
β₂, \ldots, βₚ que minimicen la suma de los errores cuadráticos entre
los valores observados y los valores predichos por el modelo.

Si trasladamos estos conceptos a nuestro estudio, la relación entre la
altitud y el número de estelas prehistóricas diseminadas por la
Península Ibérica rigiéndonos mediante los métodos de regresión lineal,
conllevarían: 1) Un primer paso, consistente en la recopilación de
datos, para lo que hemos reunido datos sobre la altitud y el número de
estelas prehistóricas en diferentes ubicaciones de la Península Ibérica,
consultando la bibliografía pertinente, así como una base de datos
dedicada a las Estelas del Suroeste de la P.I en las que aparecen
georreferenciadas todas y cada una de ellas. Pero sin duda, el soporte
básico de nuestro trabajo es una cartografía realizada por Sebastián
Celestino, que reúne de manera actualizada la ubicación de las
principales estelas halladas en la península. 2) Un segundo paso, que
conlleva la preparación de dichos datos, organizándolos en un formato
adecuado para el análisis de regresión lineal, y asegurándonos de que
las observaciones de altitud y número de estelas correspondan
adecuadamente. Con este fin, creamos una pequeña base de datos en Excel
que servirá como base para importar la información recopilada en R. 3)
Un tercer paso, que es el análisis propiamente dicho, aplicando las
fórmulas y los conceptos de regresión lineal previamente citados. 4) Un
último paso, donde expondremos la interpretación de los resultados de
nuestro análisis y llegaremos a unas conclusiones concretas.

\hypertarget{desarrollo}{%
\subsubsection{4. DESARROLLO:}\label{desarrollo}}

A continuación,

\hypertarget{uxba-cargamos-la-base-de-datos}{%
\subsection{1º) Cargamos la Base de
Datos:}\label{uxba-cargamos-la-base-de-datos}}

\begin{Shaded}
\begin{Highlighting}[]
\NormalTok{dat }\OtherTok{\textless{}{-}} \FunctionTok{read\_excel}\NormalTok{(}\AttributeTok{path =} \StringTok{"Base de datos Estelas.xlsx"}\NormalTok{)}
\FunctionTok{colnames}\NormalTok{(dat) }\OtherTok{\textless{}{-}} \FunctionTok{c}\NormalTok{(}\StringTok{"ID"}\NormalTok{, }\StringTok{"Altitud"}\NormalTok{, }\StringTok{"Nestelas"}\NormalTok{) }\CommentTok{\# cambiamos el nombre de las columnas}
\end{Highlighting}
\end{Shaded}

\hypertarget{uxba-comprobamos-la-correlaciuxf3n-entre-las-variables}{%
\subsection{2º) Comprobamos la correlación entre las
variables:}\label{uxba-comprobamos-la-correlaciuxf3n-entre-las-variables}}

\begin{Shaded}
\begin{Highlighting}[]
\FunctionTok{cor.test}\NormalTok{(dat}\SpecialCharTok{$}\NormalTok{Altitud, dat}\SpecialCharTok{$}\NormalTok{Nestelas) }\CommentTok{\# este paso nos informa sobre las posibilidades de que el valor aportado sea debido al azar. En nuestro caso, observamos una correlación muy significativa con un coeficiente de correlacion de {-}0.78. }
\end{Highlighting}
\end{Shaded}

\begin{verbatim}
## 
##  Pearson's product-moment correlation
## 
## data:  dat$Altitud and dat$Nestelas
## t = -5.2148, df = 17, p-value = 7.009e-05
## alternative hypothesis: true correlation is not equal to 0
## 95 percent confidence interval:
##  -0.9132523 -0.5130022
## sample estimates:
##        cor 
## -0.7844309
\end{verbatim}

\begin{Shaded}
\begin{Highlighting}[]
\FunctionTok{ggplot}\NormalTok{(}\AttributeTok{data =}\NormalTok{ dat, }\FunctionTok{aes}\NormalTok{(}\AttributeTok{x=}\NormalTok{Altitud, }\AttributeTok{y=}\NormalTok{Nestelas)) }\SpecialCharTok{+} \FunctionTok{geom\_point}\NormalTok{()}
\end{Highlighting}
\end{Shaded}

\includegraphics{Práctica_Final_files/figure-latex/corr-1.pdf}

\hypertarget{uxba-graficamos-la-distribucion-del-nuxba-de-estelas}{%
\subsection{3º) Graficamos la distribucion del nº de
estelas:}\label{uxba-graficamos-la-distribucion-del-nuxba-de-estelas}}

\begin{Shaded}
\begin{Highlighting}[]
\FunctionTok{ggplot}\NormalTok{(dat, }\FunctionTok{aes}\NormalTok{(}\AttributeTok{x=}\NormalTok{Nestelas)) }\SpecialCharTok{+} \FunctionTok{geom\_bar}\NormalTok{()}
\end{Highlighting}
\end{Shaded}

\includegraphics{Práctica_Final_files/figure-latex/norm-1.pdf}

\begin{Shaded}
\begin{Highlighting}[]
\FunctionTok{shapiro.test}\NormalTok{(dat}\SpecialCharTok{$}\NormalTok{Nestelas)}
\end{Highlighting}
\end{Shaded}

\begin{verbatim}
## 
##  Shapiro-Wilk normality test
## 
## data:  dat$Nestelas
## W = 0.87693, p-value = 0.01899
\end{verbatim}

\hypertarget{uxba-exponemos-nuestro-conjunto-de-datos-en-una-tabla}{%
\subsection{4º) Exponemos nuestro conjunto de datos en una
tabla:}\label{uxba-exponemos-nuestro-conjunto-de-datos-en-una-tabla}}

\begin{Shaded}
\begin{Highlighting}[]
\NormalTok{dat}\SpecialCharTok{$}\NormalTok{x2 }\OtherTok{\textless{}{-}}\NormalTok{ dat}\SpecialCharTok{$}\NormalTok{Altitud}\SpecialCharTok{*}\NormalTok{dat}\SpecialCharTok{$}\NormalTok{Altitud}
\NormalTok{dat}\SpecialCharTok{$}\NormalTok{xy }\OtherTok{\textless{}{-}}\NormalTok{ dat}\SpecialCharTok{$}\NormalTok{Nestelas}\SpecialCharTok{*}\NormalTok{dat}\SpecialCharTok{$}\NormalTok{Altitud}
\NormalTok{dat }\CommentTok{\#Visualizamos nuestra tabla de datos}
\end{Highlighting}
\end{Shaded}

\begin{verbatim}
## # A tibble: 19 x 5
##       ID Altitud Nestelas      x2    xy
##    <dbl>   <dbl>    <dbl>   <dbl> <dbl>
##  1     1      50        6    2500   300
##  2     2     100       18   10000  1800
##  3     3     200        9   40000  1800
##  4     4     300       21   90000  6300
##  5     5     400       14  160000  5600
##  6     6     500       12  250000  6000
##  7     7     600       10  360000  6000
##  8     8     700        6  490000  4200
##  9     9     800        1  640000   800
## 10    10     900        6  810000  5400
## 11    11    1000        5 1000000  5000
## 12    12    1100        4 1210000  4400
## 13    13    1200        4 1440000  4800
## 14    14    1300        2 1690000  2600
## 15    15    1400        3 1960000  4200
## 16    16    1500        2 2250000  3000
## 17    17    1600        1 2560000  1600
## 18    18    1700        1 2890000  1700
## 19    19    1800        0 3240000     0
\end{verbatim}

El método de mínimos cuadrados ordinarios calcula la pendiente (B1) y el
intercepto (B0) minimizando el cuadrado de las distancias de los puntos
x, y en relación con la recta de regresión \[y_1 = B_0 + x_1 + e_1\]

\hypertarget{uxba-generamos-el-modelo-lineal-para-comprobar-si-la-altitud-predice-el-nuxfamero-de-estelas}{%
\subsection{5º) Generamos el modelo lineal para comprobar si la altitud
predice el número de
estelas:}\label{uxba-generamos-el-modelo-lineal-para-comprobar-si-la-altitud-predice-el-nuxfamero-de-estelas}}

\begin{Shaded}
\begin{Highlighting}[]
\NormalTok{modelo }\OtherTok{\textless{}{-}} \FunctionTok{lm}\NormalTok{(}\AttributeTok{formula =}\NormalTok{ Nestelas }\SpecialCharTok{\textasciitilde{}}\NormalTok{ Altitud,}\AttributeTok{data =}\NormalTok{ dat)}
\FunctionTok{summary}\NormalTok{(modelo)}
\end{Highlighting}
\end{Shaded}

\begin{verbatim}
## 
## Call:
## lm(formula = Nestelas ~ Altitud, data = dat)
## 
## Residuals:
##     Min      1Q  Median      3Q     Max 
## -7.7692 -1.0712  0.3019  1.0669  9.3391 
## 
## Coefficients:
##              Estimate Std. Error t value Pr(>|t|)    
## (Intercept) 14.190814   1.703852   8.329 2.10e-07 ***
## Altitud     -0.008433   0.001617  -5.215 7.01e-05 ***
## ---
## Signif. codes:  0 '***' 0.001 '**' 0.01 '*' 0.05 '.' 0.1 ' ' 1
## 
## Residual standard error: 3.831 on 17 degrees of freedom
## Multiple R-squared:  0.6153, Adjusted R-squared:  0.5927 
## F-statistic: 27.19 on 1 and 17 DF,  p-value: 7.009e-05
\end{verbatim}

\begin{Shaded}
\begin{Highlighting}[]
\CommentTok{\#Con un p{-}valor de 0.00007, podemos confirmar que sí existe una fuerte relación entre la altitud y el número de estelas. En cuanto a la significancia, tenemos un valor de 0 en los tres asteriscos (máxima confianza), lo que quiere decir que, aunque no tenemos pruebas irrefutables de que la linealidad sea exacta, conlos datos que los que disponemos no podemos evidenciar lo contrario}

\CommentTok{\#Guardamos los coeficientes}
\NormalTok{coeficientes }\OtherTok{\textless{}{-}} \FunctionTok{coef}\NormalTok{(modelo)}
\NormalTok{coeficientes}
\end{Highlighting}
\end{Shaded}

\begin{verbatim}
##  (Intercept)      Altitud 
## 14.190814461 -0.008432972
\end{verbatim}

\hypertarget{uxba-generamos-un-gruxe1fico-de-dispersiuxf3n}{%
\subsection{6º) Generamos un gráfico de
dispersión:}\label{uxba-generamos-un-gruxe1fico-de-dispersiuxf3n}}

\begin{Shaded}
\begin{Highlighting}[]
\FunctionTok{ggplot}\NormalTok{(dat, }\FunctionTok{aes}\NormalTok{(Altitud, Nestelas)) }\SpecialCharTok{+}
  \FunctionTok{geom\_point}\NormalTok{() }\SpecialCharTok{+}                   \CommentTok{\# Agregar la nube de puntos}
  \FunctionTok{geom\_abline}\NormalTok{(}\AttributeTok{intercept =}\NormalTok{ coeficientes[}\DecValTok{1}\NormalTok{], }\AttributeTok{slope =}\NormalTok{ coeficientes[}\DecValTok{2}\NormalTok{], }\AttributeTok{color =} \StringTok{"red"}\NormalTok{) }\SpecialCharTok{+} \CommentTok{\# Agregar la recta de regresión}
  \FunctionTok{labs}\NormalTok{(}\AttributeTok{title=} \StringTok{"Gráfico de dispersión"}\NormalTok{,}\AttributeTok{caption =} \FunctionTok{paste}\NormalTok{(}\StringTok{"Recta de regresión: Nestelas = "}\NormalTok{, }\FunctionTok{round}\NormalTok{(coeficientes[}\DecValTok{1}\NormalTok{],}\DecValTok{2}\NormalTok{), }\StringTok{" + "}\NormalTok{, }\FunctionTok{round}\NormalTok{(coeficientes[}\DecValTok{2}\NormalTok{],}\DecValTok{2}\NormalTok{), }\StringTok{"altitud"}\NormalTok{),}
       \AttributeTok{x =} \StringTok{"Altitud"}\NormalTok{, }\AttributeTok{y =} \StringTok{"Nº Estelas"}\NormalTok{) }\CommentTok{\# Editamos los nombres de los ejes}
\end{Highlighting}
\end{Shaded}

\includegraphics{Práctica_Final_files/figure-latex/disp-1.pdf}

\hypertarget{uxba-observamos-las-medidas-de-influencia}{%
\subsection{7º) Observamos las medidas de
influencia:}\label{uxba-observamos-las-medidas-de-influencia}}

\begin{Shaded}
\begin{Highlighting}[]
\NormalTok{medidas\_influencia }\OtherTok{\textless{}{-}} \FunctionTok{influence.measures}\NormalTok{(}\FunctionTok{lm}\NormalTok{(Nestelas }\SpecialCharTok{\textasciitilde{}}\NormalTok{ Altitud, dat))}
\NormalTok{medidas\_influencia}
\end{Highlighting}
\end{Shaded}

\begin{verbatim}
## Influence measures of
##   lm(formula = Nestelas ~ Altitud, data = dat) :
## 
##       dfb.1_  dfb.Altt    dffit cov.r   cook.d    hat inf
## 1  -1.223114  1.031730 -1.22351 0.685 5.60e-01 0.1822   *
## 2   0.610876 -0.506538  0.61174 1.088 1.78e-01 0.1674    
## 3  -0.396055  0.315408 -0.39875 1.167 7.96e-02 0.1406    
## 4   1.159311 -0.877001  1.18098 0.467 4.48e-01 0.1173   *
## 5   0.274255 -0.193893  0.28557 1.141 4.14e-02 0.0976    
## 6   0.148745 -0.095780  0.16090 1.185 1.35e-02 0.0815    
## 7   0.053922 -0.030241  0.06216 1.205 2.05e-03 0.0690    
## 8  -0.119427  0.053310 -0.15260 1.148 1.21e-02 0.0599    
## 9  -0.295593  0.082376 -0.44394 0.811 8.63e-02 0.0545    
## 10 -0.019183  0.000179 -0.03689 1.188 7.22e-04 0.0526    
## 11 -0.016890 -0.008351 -0.04735 1.188 1.19e-03 0.0543    
## 12 -0.011587 -0.020553 -0.06021 1.192 1.92e-03 0.0596    
## 13 -0.000209 -0.002431 -0.00506 1.212 1.36e-05 0.0684    
## 14  0.008606 -0.056928 -0.09645 1.212 4.91e-03 0.0808    
## 15 -0.010621  0.036248  0.05369 1.246 1.53e-03 0.0967    
## 16 -0.012843  0.033153  0.04482 1.275 1.07e-03 0.1162    
## 17 -0.011892  0.026155  0.03316 1.310 5.84e-04 0.1393    
## 18 -0.059297  0.117399  0.14208 1.336 1.07e-02 0.1659    
## 19 -0.064356  0.118242  0.13824 1.391 1.01e-02 0.1961   *
\end{verbatim}

\begin{Shaded}
\begin{Highlighting}[]
\FunctionTok{plot}\NormalTok{(}\FunctionTok{cooks.distance}\NormalTok{(modelo), }\AttributeTok{pch =} \DecValTok{20}\NormalTok{, }\AttributeTok{main =} \StringTok{"Valores de Cook\textquotesingle{}s distance"}\NormalTok{)}
\end{Highlighting}
\end{Shaded}

\includegraphics{Práctica_Final_files/figure-latex/inf-1.pdf}

\begin{Shaded}
\begin{Highlighting}[]
\CommentTok{\#Los valores de Cook\textquotesingle{}s distance se utilizan en regresión lineal para evaluar la influencia de cada observación individual en el ajuste del modelo. Esta métrica ayuda a identificar observaciones atípicas o influyentes que pueden tener un impacto significativo en los resultados del modelo.}
\end{Highlighting}
\end{Shaded}

\hypertarget{uxba-realizamos-el-supuesto-de-independencia-de-los-residuos}{%
\subsection{8º) Realizamos el supuesto de independencia de los
residuos:}\label{uxba-realizamos-el-supuesto-de-independencia-de-los-residuos}}

\begin{Shaded}
\begin{Highlighting}[]
\FunctionTok{dwtest}\NormalTok{(modelo)}
\end{Highlighting}
\end{Shaded}

\begin{verbatim}
## 
##  Durbin-Watson test
## 
## data:  modelo
## DW = 1.98, p-value = 0.3778
## alternative hypothesis: true autocorrelation is greater than 0
\end{verbatim}

\hypertarget{uxba-confirmamos-que-los-errores-del-modelo-permanecen-constantes-para-todo-el-rango-de-estimaciones}{%
\subsection{9º) Confirmamos que los errores del modelo permanecen
constantes para todo el rango de
estimaciones:}\label{uxba-confirmamos-que-los-errores-del-modelo-permanecen-constantes-para-todo-el-rango-de-estimaciones}}

\begin{Shaded}
\begin{Highlighting}[]
\FunctionTok{plot}\NormalTok{(modelo}\SpecialCharTok{$}\NormalTok{fitted.values, modelo}\SpecialCharTok{$}\NormalTok{residuals, }\AttributeTok{xlab =} \StringTok{"Valores ajustados"}\NormalTok{, }\AttributeTok{ylab =} \StringTok{"Residuos"}\NormalTok{)}
\end{Highlighting}
\end{Shaded}

\includegraphics{Práctica_Final_files/figure-latex/mod_err-1.pdf}

\hypertarget{conclusiones}{%
\subsubsection{5. CONCLUSIONES:}\label{conclusiones}}

Desarollado nuestro análisis, procedemos a exponer nuestras
interpretaciones a modo de conclusión:

\begin{itemize}
\tightlist
\item
  La correlación de nuestro modelo es una muy marcada y eminentemente
  signficativa, con un valor p de -0,78, lo que quiere decir que por
  cada unidad que aumentamos en altitud, disminuye en 0.78 el número de
  estelas.
\item
  La distribución que muestra nuestro conjunto de datos podría
  asemejarse a una distribución de tipo normal, pero a falta de una
  mayor cantidad de datos no podemos atribuila a una como tal.
\item
  Cuanto más aumenta la altitud, más se ajusta nuestro modelo a la
  distribución normal, si bien los valores más bajos en altitud parecen
  representar una mayor dispersión de las estelas.
\item
  Al comprobar el supuesto de independencia de los residuos, comprobamos
  que estos son independientes, y que los errores del modelo en su
  mayoría son muy próximos a 0, que es lo deseado.
\end{itemize}

Por tanto, y sin evidencias que parezcan indicar lo contrario, podríamos
afirmar que la teoría de que la presencia de las estelas del suroeste
viene determinada por la geografía ibérica encaja con nuestras
conclusiones. Podemos añadir, además, que es lógico que estos monumentos
se sitúen en lugares de baja altitud, debido al coste que debe suponer
trasladar materia prima a lugares menos accesibles, como suelen ser los
puertos de montaña y las zonas muy elevadas. Esto además concuerda con
la idea de que las estelas pudieron haber constituido marcadores
paisajísticos de relevancia y de orientación, señalando cruces de
caminos y vados de ríos. Así pues, y fuera aparte de las posibles
interpretaciones simbólicas que de ellas puedan extraerse, con nuestro
trabajo parece quedar claro que, con seguridad, el patrón geográfico de
estos megalitos es un hecho.

\hypertarget{bibliografuxeda}{%
\subsubsection{BIBLIOGRAFÍA:}\label{bibliografuxeda}}

\begin{itemize}
\tightlist
\item
  CARRERO-PAZOS, M. (2018): ``Modelando dinámicas de movilidad y
  visibilidad en los paisajes megalíticos gallegos. El caso del Monte de
  Santa Mariña y su entorno (Comarca de Sarria, Lugo)'', Trabajos de
  Prehistoria, 75 (2), 287-306.
\item
  CELESTINO PÉREZ, S y PANIEGO DÍAZ, P. (2021). Últimas investigaciones
  sobre las estelas de guerrero y diademadas de la península Ibérica.
  Paleohispánica. Revista sobre lenguas y culturas de la Hispania
  Antigua. 21. 71-93.
\item
  DÍAZ-GUARDAMINO, M. (2010). Las estelas decoradas en la Prehistoria de
  la Península Ibérica. Tesis Doctoral, Universidad Complutense de
  Madrid.
\item
  GALÁN DOMINGO, E., MARTÍN BRAVO, A. M. (1992): ``Megalitismo y zonas
  de paso en la cuenca extremeña del Tajo'', Zephyrus 44-45, 193-205.
\item
  MORAL PELÁEZ, I. (2016). Modelos de regresión: lineal simple y
  regresión logística. Revista Seden, 14, 195-214.
\item
  RIVERA JIMÉNEZ, T et al.~(2021): ``The Cañaveral de León stela
  (Huelva, Spain). A monumental sculpture in a landscape of settlements
  and pathways''. Journal of Archaeological Science:Reports, vol.~40,
  p.~103-251.
\item
  Base de datos: \url{http://www.estelasdecoradas.es/index.php}
\end{itemize}

\end{document}
